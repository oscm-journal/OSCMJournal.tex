\documentclass[preprint]{oscmjournal}
\usepackage[utf8]{inputenc}
% \usepackage[english]{babel}
\usepackage{amssymb}
\usepackage{amsmath,amsfonts}
\usepackage{amsthm}
\usepackage{algorithmic}
\usepackage{algorithm}
\usepackage{array}
\usepackage{hyperref}
\usepackage{balance}
\urlstyle{same}
\usepackage{xurl}
\usepackage[caption=false,font=normalsize,labelfont=sf,textfont=sf]{subfig}
\usepackage{lineno}
\usepackage{textcomp}
\usepackage{stfloats}
\usepackage{url}
\usepackage{verbatim}
\usepackage{graphicx}
\usepackage{multirow}
\usepackage{booktabs,siunitx,array,threeparttable}
\sisetup{group-minimum-digits=4}

\newtheorem{theorem}{Theorem}
\theoremstyle{remark}
\newtheorem{remark}{Remark}


\usepackage[style=apa]{biblatex}
\addbibresource{ref.bib}

% \linenumbers


\title{
A Robust and Efficient Optimization Model for Electric Vehicle Charging Stations in Developing Countries under Electricity Uncertainty
}
\authorinfo{
    \textbf{Author 1}\\
    University of Tennessee, Knoxville\\
    Haslam College of Business\\
    Email: email@email.com\\
    \vspace{1em}
    \textbf{Author 2}\\
    University of Kansas, Knoxville\\
    Something College of Business\\
    Email: email2@email.com\\
    \vspace{1em}
    \textbf{Author 3}\\
    University of Kansas, Knoxville\\
    Something College of Business\\
    Email: email2@email.com
}

\keywords{omnichannel, consumer, retailing, review, bibliometric}

\begin{document}

\maketitle

\begin{abstract}
    The rising demand for electric vehicles (EVs) worldwide necessitates the development of robust and accessible charging infrastructure, particularly in developing countries where electricity disruptions pose a significant challenge. Earlier charging infrastructure optimization studies do not rigorously address such service disruption characteristics, resulting in suboptimal infrastructure designs. To address this issue, we propose an efficient simulation-based optimization model that estimates candidate stations' service reliability and incorporates it into the objective function and constraints. We employ the control variates (CV) variance reduction technique to enhance simulation efficiency. Our model provides a highly robust solution that buffers against uncertain electricity disruptions, even when candidate station service reliability is subject to underestimation or overestimation. Using a dataset from Surabaya, Indonesia, our numerical experiment demonstrates that the proposed model achieves a 13\% higher average objective value compared to the non-robust solution. Furthermore, the CV technique successfully reduces the simulation sample size up to 10 times compared to Monte Carlo, allowing the model to solve efficiently using a standard MIP solver. Our study provides a robust and efficient solution for designing EV charging infrastructure that can thrive even in developing countries with uncertain electricity disruptions.
\end{abstract}


\section{INTRODUCTION}
\label{sec:intro}

The growing global demand for electric vehicles (EVs) has brought to the forefront the need for reliable and easily accessible EV charging infrastructure. According to a report by the International Energy Agency, as numerous governments set ambitious goals for electrifying their transportation systems, the worldwide EV demand has exponentiated in recent years. In 2010, there were only approximately 17,000 EVs on the world’s roads. In 2019, for instance, China led the global EV market, with more than 1 million EVs cars sold that year (more than 50\% of global EV demand), followed by the whole of Europe with 561,000 cars sold and the USA with 327,000 cars sold. This trend is projected to persist in the upcoming years \autocite{evoutlook}.


Developing countries are also striving to promote EV adoption, coupled with greener electricity \autocite{lai2022critical} to expedite the achievement of their sustainability goals. For example, Indonesia has set an ambitious target of having 20\% of all automobile sales be electric by 2025, with a long-term goal of achieving fully electrified transportation by 2050 \autocite{indonesiaenergy}. However, developing countries like Indonesia face significant infrastructure constraints that must be addressed to achieve these goals. The availability of EV charging infrastructure is a crucial issue that must be addressed to support the widespread adoption of EVs. In Indonesia, there were only 240 public EV charging points across the country as of 2021 \autocite{ev_240}. However, an estimated 31,000 EV charging stations are required throughout the country to support sustainable electrification of vehicles in the country \autocite{ev_31000}. 

This lacking infrastructure issue is not unique to Indonesia and is faced by many other developing countries to support the growth of EV adoption. Tackling this challenge by designing a convenient and reliable EV charging network is, however, a very complex task. To ensure a convenient location, it is essential to consider factors such as population density or potential EV demand distribution \autocite{khalid2019comprehensive}. However, in major cities in developing countries, finding suitable land for charging stations may be challenging due to limited space availability. Furthermore, in developing countries, service uncertainty, including electricity, is one of the most significant issues. Implementing smart charging strategies \autocite{fachrizal2020smart} becomes hardly feasible due to electricity supply uncertainty. Outages and other electricity disruptions often occur, posing a significant problem for users who demand reliable service. 


To address this challenge, our study proposes a robust solution for designing EV  charging infrastructure that accounts for the challenge of electricity disruptions in developing countries. We introduce a simulation-based optimization model that estimates the service reliability of candidate charging stations and incorporates this information into the objective function and constraints. This approach offers a versatile solution by utilizing simulation approaches compared to previous works that assume available disruption probability models. Additionally, we employ a variance reduction technique called control variates (CV) to enhance simulation efficiency, reducing the required sample size by up to 10 times compared to naive Monte Carlo (MC) simulations. This results in an efficient mixed-integer programming (MIP) model that solves for optimal solutions that strike the balanced objective between minimizing the total cost of operating and investing in the charging infrastructure and providing high-quality service to the public. Fig. \ref{fig:framework} illustrates the comparison between the traditional modeling approach without variance reduction vs. the proposed framework that utilizes the variance reduction technique to achieve a tighter confidence interval (hence much more precise output) with less computational burden.

\begin{figure}
    \centering
    \includegraphics[width=0.5\linewidth]{figs/oscm.png}
    \caption{Illustration of the proposed framework. The incorporation of partial knowledge about real-world uncertainty in the modeling allows us to reduce the variance of the simulation outputs, hence reducing the required sample size to achieve a tighter confidence interval (i.e., higher confidence results).}
    \label{fig:framework}
\end{figure}

 
 Our work contributes in three key ways. Firstly, we propose a model that specifically addresses the critical issue of electricity disruption in EV charging station planning, particularly in developing countries. Secondly, we integrate the estimation of disruption probabilities into our model, providing a more data-driven approach compared to previous works that assumed available disruption probability models apriori. Finally, our study demonstrates the robustness of the proposed model in solving EV charging infrastructure problems by comparing its performance to a non-robust model, even when disruption probabilities are slightly under or over-estimated. Our numerical experiment, based on an EV dataset from Surabaya, Indonesia, shows that our model achieves a 13\% higher average objective value compared to the non-robust solution, highlighting its superior performance to help build sustainable and thriving ecosystems for EVs, both in developed and developing countries in the years to come.

 

The rest of this paper is structured as follows. In Section \ref{sec:lit_review}, we provide a concise overview of the literature related to the optimization of EV charging infrastructure% and highlight the need to address electricity disruptions that characterizes developing countries. 
We then present the proposed model formulations in Section \ref{sec:problem_formulation} %, where we derive the stochastic model considering electricity disruptions 
and approach incorporating the CV technique to estimate the service reliability (i.e. the complement of disruption probability). In Section \ref{sec:exp}, we describe the experiment settings and discuss the main findings in Section \ref{sec:discussion}. Finally, we conclude our work in Section \ref{sec:conclusion}.


\section{LITERATURE REVIEW}
\label{sec:lit_review}

In this section, we briefly review earlier works directly related to the planning of EV charging infrastructure and relevant case studies that motivate our approach.  Examining these earlier works offers insight into the evolution of methodologies, leading to the proposed work, which uniquely introduces a combination of stochastic modeling and variance reduction techniques. The summary is provided in Table \ref{tab:review}.

\begin{table*}
    % \sffamily\small
    \centering
    \caption{Summary of EV Charging Infrastructure Optimization Model}    
    \resizebox{0.7\textwidth}{!}{
        \begin{threeparttable}[t]
            \centering
            \begin{tabular}{@{}llcc@{}}
\toprule
\multicolumn{1}{c}{\textbf{Authors}} & \multicolumn{1}{c}{\textbf{Objectives}} & \textbf{Modeling$^*$} & \textbf{VR$^\dagger$} \\ \midrule
\textcite{he2016sustainability} & Minimize station count & D & - \\
\textcite{albana2022optimal} & Minimize development and transport costs & D & - \\
\textcite{ko2017determining} & Maximize service demand & D & - \\
\textcite{jegham2020locating} & Minimize infrastructure and access costs & D & - \\
\textcite{asamer2016optimizing} & Maximize covered taxi trips & D & - \\
\textcite{frade2011optimal} & Maximize demand covered & D & - \\
\textcite{kunith2017electrification} & Minimize battery and infrastructure costs & D & - \\
\textcite{spieker2017multi} & Maximize POI and traffic coverage & D & - \\
\textcite{fekete2016improved} & Maximize covered demand volume & D & - \\
\textcite{yuan2015competitive} & Minimize station count & D & - \\
\textcite{hua2019optimal} & Maximize service performance & D & - \\
\textcite{fredriksson2021optimal} & Minimize station count & D & - \\
\textcite{miljanic2018efficient} & Minimize path length & D & - \\
\textcite{bouguerra2019determining} & Minimize station count & D & - \\
\textcite{lo2020electric} & Minimize station count & D & - \\
\textcite{sun2020locating} & Maximize total EV flow & D & - \\
\textcite{huang2023geographic} & Minimize station, construction, and travel costs & D & - \\
\textcite{hamed2023random} & Maximize EV coverage in day and night & D & - \\
\textcite{zhang2022bi} & Minimize operating costs & D & - \\
\textcite{amilia2022designing} & Minimize total cost & D & - \\
\textcite{hosseini2015refueling} & Minimize station, recharging, and travel costs & S & - \\
\textcite{yildiz2019urban} & Minimize infrastructure costs & S & - \\
\textcite{alhazmi2017optimal} & Maximize station coverage & S & - \\
\textcite{li2016multi} & Minimize total cost & S & - \\
\textcite{ren2019location} & Maximize EV coverage by stations & S & - \\
\textcite{vazifeh2019optimizing} & Minimize station count & S & - \\
\textcite{speth2022public} & Maximize traffic volume and vehicles served & S & - \\
\textbf{This study} & \textbf{Maximize service-penalized profit} & \textbf{S} & \textbf{CV}$^\ddagger$ \\ 
\bottomrule
\end{tabular}
\begin{tablenotes}
    \item[~]$^*$\textit{Modeling: D (Deterministic), S (Stochastic)}
    \item[~]$^\dagger$\textit{VR: Variance Reduction}
    \item[~]$^\ddagger$\textit{CV: Control Variates}
\end{tablenotes}
\end{threeparttable}%
}
        %\vspace{-1em}        
    \label{tab:review}
    % \normalsize\normalfont
\end{table*}


The planning of EV charging infrastructure can be viewed as a facility location problem, which aims to minimize an objective function subject to constraints related to the desired performance of the network facilities. Early studies, including those by \autocite{he2016sustainability} and \autocite{albana2022optimal}, adopted deterministic models focusing on minimizing charging stations and development costs, respectively. \autocite{ko2017determining} sought to maximize service demand, whereas \autocite{jegham2020locating} aimed to minimize infrastructure and access costs. Similar objectives were pursued by \autocite{asamer2016optimizing}, \autocite{frade2011optimal}, and \autocite{kunith2017electrification}, with deterministic models being the common methodology.

Several other studies, like those conducted by \textcite{spieker2017multi}, \textcite{fekete2016improved}, \textcite{yuan2015competitive}, \textcite{hua2019optimal}, \textcite{fredriksson2021optimal}, and \textcite{miljanic2018efficient}, continued the trend of deterministic models, exploring various aspects of EV charging station optimization. Other researchers, including \textcite{bouguerra2019determining}, \textcite{lo2020electric}, \textcite{sun2020locating},\textcite{huang2023geographic}, \textcite{hamed2023random}, \textcite{zhang2022bi}, and \textcite{amilia2022designing}, focused on minimizing the number of charging stations or the operating cost, or maximizing the EV flow coverage.

Another line of work integrates charging infrastructure into the smart-grid design \autocite{chen2012iems, lam2012multi, lam2012capacity} or other renewable energy sources such as solar cells \autocite{guo2012study}. While this approach provides an integrated solution to renewable energy issues and amplifies the positive impact of EVs on the environment, it may not be practical for urban areas in developing countries. A comprehensive review of charging infrastructure designs is presented by \textcite{unterluggauer2022electric}, emphasizing the need for increasingly detailed modeling that accounts for randomness and variability. However, there is a lack of rigorous real-world case studies that emphasize uncertainty quantification in the modeling framework.

Several case studies have been conducted in both developed and developing countries. For example, \textcite{frade2011optimal} studied the problem of slow-charging technology in Lisbon, where vehicles are often parked overnight. In contrast, \textcite{huang2016design} considered both fast- and slow-charging technologies, focusing on robustly covering all demands and avoiding partial fulfillment in the city of Toronto. Another case study was conducted by \textcite{erbacs2018optimal} using a GIS-based model in Ankara and adopting a fuzzy approach. A city-scale simulation was developed for Singapore by \textcite{bi2017simulation}, focusing on the trade-off between cost minimization and customer accessibility maximization. Lastly, \textcite{amilia2022designing} proposed a set covering model for EV charging stations in Surabaya but ignored electricity disruption and only provided redundant demand coverage to provide a buffer against uncertainty, resulting in an overly simplified model and sub-optimal solutions.

In light of these studies, it is clear that the EV facility location problem is a complex and multifaceted issue that requires a tailored approach for different regions and contexts. Developing countries, in particular, may face unique challenges, such as power electricity disruptions, that must be considered in the planning and design of EV facilities. Such disruptions and uncertainty are addressed only in a handful of studies. For instance, \textcite{zhang2023optimal} uses a multi-criteria decision-making approach aiming to strike a balanced solution against flooding disruption that maximizes the charging convenience, minimizes the impact of flood hazards, and minimizes the impact of existing charging stations using TOPSIS. \textcite{liu2022optimal} integrates the electric bus charging stations with photovoltaic and energy storage systems using a two-stage stochastic programming model, enabling them to incorporate the uncertainty of PV power outputs. \textcite{hussain2020optimal} optimizes the size of the energy storage system considering the annualized cost, penalty cost for buying power during peak hours, and penalty cost for resilience violations. Other works that consider stochastic modeling include \textcite{hosseini2015refueling, yildiz2019urban, alhazmi2017optimal, li2016multi, ren2019location, vazifeh2019optimizing, speth2022public}, which directly use either structure of the stochastic models or simulations to represent elements of uncertainty into their optimization models. The caveat is that the resulting model can be extremely hard to solve, especially when a solution with high confidence is desired.


The proposed work extends the use of stochastic modeling and introduces control variates \autocite{hesterberg1998control}, a variance reduction technique that can speed up a simulation-based optimization model, to the field. We propose an approach that addresses the challenges of the need to account for electricity disruptions via simulation and controlling the resulting objective value uncertainties by adjusting the simulation sample size. Simulation modeling enables the modeler to adjust the degree of modeling fidelity, depending on the prior knowledge available, and can be easily verified by estimating the probability of electricity disruptions and comparing it with available historical data. The resulting simulation-based robust model can be accelerated using variance reduction techniques (i.e., control variates), and it offers a more accurate and practical approach for planning and designing EV charging infrastructure that considers uncertainty and disruptions. The integration of stochastic modeling and control variates sets this work apart from previous research, potentially paving the way for more efficient and effective EV charging station location optimization solutions.





\begin{figure}
    \centering
    \includegraphics[width=0.5\linewidth]{figs/oscm.png}
    \caption{The distribution of EV charging demands across the city of Surabaya (blue nodes) and the charging station candidates (red nodes).}
    \label{fig:candidate_location}
\end{figure}



\section{MODEL FORMULATION}
\label{sec:problem_formulation}

In this section, we describe our modeling components, including the decision variables, objective function, constraint set, model benchmarks (robust and non-robust model), and the CV method we employ to improve simulation efficiency.

\subsection{Decision Variables}
 We consider a set of demand nodes $I$ and supply nodes $J$, representing sub-district centers and charging station candidate locations in the region under study. We also consider $K$ vehicle types, representing different vehicle modalities that the residents use for commuting (here, we consider two modalities: electric motorcycles and electric cars). The average time to travel from node $i \in I$ to node $j \in J$ is denoted by $d_{ij}$. A threshold parameter $d_{max}$ is introduced as an upper bound for this travel time as a proxy to study the robustness of the solution w.r.t. consumer time-to-travel for charging. 

The decision variables include binary variables
$x_j$ indicating whether the charging station candidate $j$ is selected or not and $y_{ij}$ indicating whether demand node $i$ is to be assigned to be served by charging station $j$. In addition, we also use integer decision variables $v_{ij}^k$ and $u_j$, denoting the number of electric vehicles of type $k$ from node $i$ charged at node $j$ and the number of units of charging connectors installed at node $j$, respectively.
\begin{align}
x_j &= \begin{cases}
1, & \text{if station } j \in J \text{ is selected} \\
0, & \text{otherwise}
\end{cases} \\
y_{ij}&= \begin{cases}
1, & \text{if node } i \in I \text{ is assigned to node } j \in J\\
0, & \text{otherwise},
\end{cases} \\
v_{ij}^k &\in \{0, 1, \cdots\}, \forall i \in I, j \in J, k \in K \\
u_{j} &\in \{0, 1, \cdots\}, \forall j \in J
\end{align}

Each opened station $j$ incurs a daily cost $h_j$ and can only accommodate $q_j$ charging connectors due to limited space. Each charging connector incurs $g$ daily operational cost and has a limited daily charging throughput of $c_j$ kWh. A vehicle type $k$ takes $e_k$ kWh energy and $t_k$ time to charge using fast-charging technology. We use the electricity price denoted by $r$ to convert the energy used to monetary value. 


\subsection{Objective Function}
The objective is to maximize daily profits under random disruption events at each station, i.e., the revenue from all undisrupted stations minus operational and investment costs. We add a penalty term for any unmet customer demands due to the disruptions to study proper incentivizing mechanisms to achieve further robust models in the ablation study.

To this end, we consider each charging station $j \in J$ to have a reliability 
\begin{align}
p_j = \mathbb P(Z_j \leq z_j) = \mathbb E [\mathbb I(Z_j) \leq z_j].
\end{align}
The disruption events are simulated utilizing random variable $Z = [Z_j]_{\forall j \in J} \sim q$. $Z_j$ represents the underlying state triggering electricity disruption at station $j$ whenever it exceeds some threshold $z_j$. In practice, electricity disruption events may occur due to extreme weather, spiking demand, or fallen trees \autocite{10causespoweroutages} (in which $Z_j$ might represent wind speed, cumulative region-wide demand, or fallen tree branch weights, respectively, that hits electrical equipment and $z_j$ is the equipment threshold to deal with the corresponding random $Z_j$ realization). \textcite{khalid2019comprehensive} presents a review of how EV charging infrastructures strain the electricity grids, which, in turn, exacerbate the likelihood of electricity outages, especially in developing countries. 

With this consideration, the objective function can be formulated as follows: we have prior information about $p_j, \forall j \in J$.
\begin{align}
\max &\sum_{i \in I} \sum_{j \in J} \sum_{k \in K} 
\underbrace{r e_k p_j  v_{ij}^k}_{\text{revenue}} - \underbrace{s d_{ij} (1-p_j)  v_{ij}^k}_{\text{penalty}}  \nonumber \\
&-  \sum_{j \in J} \underbrace{(g u_j + h_j x_j)}_{\text{total cost}}.
\label{obj}
\end{align}
On the other hand, if $p_j$ is not available, then we can use simulation to estimate the following objective:
\begin{align}
\max &\sum_{i \in I} \sum_{j \in J} \sum_{k \in K}
\underbrace{ r e_k v_{ij}^k  \mathbb E \left[\mathbb I(Z_{j} \leq z_j) \right]}_{\text{revenue}} - \underbrace{s d_{ij}  v_{ij}^k \mathbb E \left[\mathbb I(Z_{j} > z_j) \right]}_{\text{penalty}}  \nonumber \\
&-  \sum_{j \in J} \underbrace{(g u_j + h_j x_j)}_{\text{total cost}},
\label{obj_est}
\end{align}
where $\mathbb I(Z_{jl} \leq z_j)$ is binary variables indicating whether the disruption occurs or not.
\begin{align}
\mathbb I (Z_{jl} \leq z_j) = \begin{cases} 1, & \text{if } Z_{jl} \leq z_j \\ 0, &\text{otherwise}\end{cases}.
\end{align}

Monte Carlo (MC) simulation is one of the most practical methods to achieve this. MC uses $n$ i.i.d. copies of the random variable to estimate the expectation. For each $j \in J$, we first generate $Z_{j1}, Z_{j2}, \cdots Z_{jn}$. We then check if the disruption event is triggered or not at the $l$-th sample and output the binary indicators $I_{jl} = \mathbb I (Z_{jl} \leq z_j)$. Then, we use the binary indicators in our final (robust) objective function:
\begin{align}
\max &\sum_{i \in I} \sum_{j \in J} \sum_{k \in K}
 \sum_{l=1}^n  \frac1n \left( \underbrace{(r e_k v_{ij}^k  I_{jl}}_{\text{revenue}}
- \underbrace{s d_{ij}  v_{ij}^k (1-I_{jl})}_{\text{penalty}} \right) \nonumber \\
&-  \underbrace{\sum_{j \in J} (g u_j + h_j x_j)}_{\text{total cost}}.
\label{obj_est_sim}
\end{align}
We call our model the \textit{Robust Model} in the experiment, to contrast with the original (\textit{Non-Robust}) model proposed by \textcite{amilia2022designing}, which is attained when setting  $I_{jl} = 1$ for all $j \in J, l \in \{1, 2, \cdots n\}$ in \eqref{obj_est_sim} during optimization. The solutions of both models are evaluated under random disruption events generated using a different random seed.


\subsection{Constraints}
The maximization of the objective function in \eqref{obj_est_sim} is subject to a set of constraints:
\begin{align}
    \text{s.t.}~&\sum_{k \in k} v_{ij}^k \leq y_{ij} M, &\forall i \in I, j \in J, \label{numberassigned}\\
    &d_{ij} y_{ij} \leq d_{max} , &\forall i \in I, j \in J, \label{distmax}\\
    &\sum_{j \in J} v_{ij}^k = w_i^k, & \forall i \in I, k \in K,
    \label{demandfulfilled}\\
    &\sum_{i \in I} \sum_{k \in K} t_k v_{ij}^k  \leq c_j u_j, & \forall j \in J, \label{charging_cap}\\
    &u_j \leq  x_j q_j, & \forall j \in J, \label{capacityopened}\\
    &\sum_{i \in I} y_{ij} \leq x_j M, & \forall j \in J, \label{assignedopened}\\
    &\sum_{j \in J} y_{ij} \geq 1, & \forall i \in I, \label{demandassigned}\\
    &\sum_{j \in J} x_j \leq N & \label{maxstations}\\
    &\sum_{j \in J} \sum_{l=1}^n \frac1n y_{ij}  I_{jl} \geq \bar{p}, & \forall i \in I \label{reliability}\\
    &\sum_{j \in J} \sum_{l=1}^n \frac1n v_{ij}^k I_{jl} \geq \sum_{j \in J} v_{ij}^k\bar{p}, & \forall i \in I, k \in K \label{reliability_v} %\\  &x_1 = 1 &
\end{align}

 In the above formulation, constraint (\ref{numberassigned}) ensures that charging stations can only charge vehicles if assigned. Constraint (\ref{distmax}) ensures the maximum time-to-charge for consumers does not exceed the set threshold $d_{max}$. Constraint (\ref{demandfulfilled}) ensures all charging demands are fulfilled, where $w_i^k$ denotes the number of vehicles of type $k$ to charge at demand point $i$. Constraint (\ref{charging_cap}) ensures that the required charging capacity to fulfill each station's assigned demand does not exceed the installed capacity. Constraint (\ref{capacityopened}) restricts the number of charging connectors installed in each station. Constraint (\ref{assignedopened}) ensures that demands are assigned only to opened stations. Constraint (\ref{demandassigned}) guarantees that at least one stations cover each demand. Constraint (\ref{maxstations}) limits the maximum number of stations to open.
 Finally, constraint (\ref{reliability}-\ref{reliability_v}) ensures that the probability that at least one of the assigned charging stations serving a given demand is not under an electricity outage is greater than or equal to $\bar{p}$, assuming that outages between stations are independent. 


\subsection{Robust vs. Non-Robust Model}

The consideration of $p_j$ in our formulation is part of our attempt to boost the robustness of the original model and address the unique challenges and characteristics of urban areas in developing countries. The Non-Robust Model ignores disruption probability, resulting in a more simplified model. Our formulation is general, in the sense that we can attain the earlier model by setting $I_{jl} = 1$ for all $j \in J, l \in \{1, 2, \cdots n\}$. This earlier model ignores disruption uncertainty and often results in an overly cost-optimized solution that can have serious performance degradation when disruption occurs. Fig \ref{fig:nrm_rm} (left) shows a non-robust solution where only two stations are selected to cover 30+ demand nodes in the city of Surabaya. In this solution, many demand nodes are only covered by one station (no redundancy), and thus, when an electricity disruption hits the charging station, the charging demands will not be met and the residents are served very poorly. Our proposed robust model aims to incorporate the disruption uncertainty and optimizes the location and capacity of EV charging stations while balancing the trade-offs between consumer service level and economic profits. This incorporation maintains a linear objective function and linearized constraints, which still yields an MIP model that can solve efficiently using standard solvers. 


\subsection{Improving the Efficiency of Disruption Probability Estimation}

While the proposed objective function in \eqref{obj_est_sim} is still linear, the sample size $n$ required to achieve high statistical confidence might blow up as the disruption probabilities $1 - p_j, \forall j \in J$ become lower (e.g., as the utilities in developing countries mature). Note that our objective essentially estimates $p_j$ by generating enough values $Z_{j1}, Z_{j2}, \cdots, Z_{jn}$, and compute
\begin{align}
\hat p_j = \frac1n \sum_{l=1}^n \mathbb I(Z_{jl} \leq z_j) 
%= 1- \frac1n \sum_{l=1}^n \mathbb I(Z_{jl} > z_j), 
\end{align}
which can be shown to be unbiased and converges to $p_j$.

\begin{remark}[Unbiased and Convergence]
\label{pf:unbiasedness}
Under the assumption that $Z = [Z_j]_{\forall j \in J} \sim q$ are independently and identically distributed, and $z_j, \forall j \in J$ are fixed threshold values,  estimator $\hat p_j$ is an unbiased and consistent estimator of $p_j$.
\end{remark}
\begin{proof}
The proof is straightforward but is provided here for completeness. 
\\
Unbiasedness:
\begin{align}
\mathbb E[\hat p_j] &= \mathbb E \left[ \frac1n \sum_{l=1}^n \mathbb I(Z_{jl} \leq z_j) \right] \\
&= \frac1n \sum_{l=1}^n \mathbb E \left[ \mathbb I(Z_{jl} \leq z_j) \right] \\
&= \frac1n \sum_{l=1}^n p_j \\
&= p_j
\end{align}
where the first equality follows from the definition of $\hat p_j$, the second equality follows from the linearity of the expectation operator to the sum of indicator functions, and the third line follows from the fact that $Z_{jl}$ are independently and identically distributed, and the third equality follows from the definition of $p_j$.
\\
Consistency:
We know that by the law of large numbers, for any $\epsilon > 0$,
\begin{align}
\lim_{n \to \infty} \mathbb P\left(\left|\hat p_j - p_j\right| \geq \epsilon\right) = 0.
\end{align}
Hence, $\hat p_j$ converges in probability to $p_j$, and thus it is a consistent estimator of $p_j$.
\end{proof}


Supposed that we already have an estimate $\hat p_j, \forall j 
\in J$. We can now plug the estimate into our optimization problem, giving
\begin{align}
\max &\sum_{i \in I} \sum_{j \in J} \sum_{k \in K}
\underbrace{r e_k \hat p_j  v_{ij}^k}_{\text{revenue}} - \underbrace{s d_{ij} (1-\hat p_j)  v_{ij}^k}_{\text{penalty}}  \nonumber \\
&-  \sum_{j \in J} \underbrace{(g u_j + h_j x_j)}_{\text{total cost}} 
\label{obj_est_phat}
\end{align}
\begin{align}
    \text{s.t.}~&\text{Constraint } \eqref{numberassigned}-\eqref{maxstations} \nonumber \\
    &\sum_{j \in J} y_{ij} \hat p_j \geq \bar{p}, & \forall i \in I \label{reliability_est_phat} \\
    &\sum_{j \in J} v_{ij}^k \hat p_j \geq \sum_{j \in J} v_{ij}^k\bar{p}, & \forall i \in I, k \in K \label{reliability_v_est_hat}.
\end{align}
Note that this formulation using $\hat p_j, \forall j \in J$ is equivalent to the robust model using indicator variables $I_{jl}, \forall j \in J, l \in \{1, 2, \cdots, n\}$ earlier that uses the objective function \eqref{obj_est_sim}.


\subsubsection{Estimating $\hat p_j$ to Sufficient Accuracy}

While $\hat p_j$ is unbiased and consistent, the sample size to ensure a precise estimate can be arbitrarily large, especially when we want a higher accuracy (e.g. when the disruption rate $1-p_j$ is tiny, such as in developed countries where utility service has high reliability). Suppose we want an $\delta$-accuracy and $1-\alpha$ confidence level to estimate $p_j = 0.9999$. Then, we can use Hoeffding's inequality to determine the sample size. According to Hoeffding's inequality, for any $\delta > 0$, the probability that the estimate deviates from the true value by more than $\delta$ is bounded by
\begin{equation}
\mathbb P(|\hat p_j - p_j| > \delta) \leq 2e^{-2n\delta^2},
\end{equation}
where $n$ is the sample size. Hence, if we want to ensure $1-\alpha$ confidence level, we set $2e^{-2n\delta^2} = \alpha$, and solve for $n$
\begin{equation}
n = \frac{1}{2\delta^2}\ln\left(\frac{2}{\alpha}\right).
\end{equation}
For instance, if we want an accuracy of $\delta = 0.0001$ and a confidence level of $1-\alpha = 0.95$, then the required sample size is
\begin{equation}
n = \frac{1}{2(0.0001)^2}\ln \left(\frac{2}{0.05}\right) \approx 114,763,
\end{equation}
which is quite huge. Figure \ref{fig:sample_size} shows the sample size (in a $\log_{10}$ scale) for various $\alpha$ and $\delta$ values. Note, however, that this is an upper bound and in practice, this sample size is not always necessary.

\begin{figure}
    \centering
    \includegraphics[width=0.5\linewidth]{figs/oscm.png}
    \caption{Sample size required to achieve $\delta$-accuracy and $1-\alpha$ confidence.}
    \label{fig:sample_size}
\end{figure}

If we have $N := |J|$ stations and each $p_j$ has to be estimated using $n\approx 114,763$ samples, then we will need $N \times 114,763$ samples to estimate the samples prior to solving the optimization problem, which can be overly burdensome if each simulation runs considers complex systems. Thus, we seek ways to improve efficiency and reduce the variance of the estimator.


\subsubsection{Improving Efficiency via Control Variates}

One way to improve the estimation efficiency and thus reduce the sample size is through the use of control variates (CV) \autocite{hesterberg1998control}. CV involves introducing a new variable that is correlated with the random variable of interest and can be easily estimated. The CV is then used to adjust the estimate of the random variable to improve its efficiency by reducing the variance of the estimator using the cheaper-to-compute random variable. In our case, we can use CV to estimate $p_j = \mathbb P(Z_j \leq z_j)$. Let $g(Z_j)$ be a function of $Z_j$ that is easy to compute. Specifically, if we consider Gaussian $q = N(\mu, \sigma)$ and $Z_j \sim q$, we can use 
\begin{align}
g(z) = \Phi(z)
\end{align}
the CDF of the standard normal distribution as the CV to compute $g(Z_j)$.  The CV estimator for $p_j$ is computed as
\begin{align}
\hat p_j = \frac{1}{n}\sum_{l=1}^n \mathbb I(Z_{jl} \leq z_j) + \pi_j \bigg( \mathbb I (X_{jl}\leq \bar z_j)-g(\bar z_j) \bigg)
\label{cv_estimator}
\end{align}
where $Z_{jl}$ is the $l$-th sample from the distribution $q$, $X_{jl}$'s are standard normal random variables  correlated with $Z_{jl}$, and $\bar z_j$ are the scaled version of $z_j$ chosen to threshold $X_{jl}$. Finally, $\pi_j$ is chosen to minimize the variance
\begin{align}
\pi_j = - \frac{\text{Cov}\bigg( \sum_{l=1}^n \mathbb I(Z_{jl}\leq z_j), \sum_{l=1}^n \mathbb I(X_{jl}\leq \bar z_j) \bigg)}{\text{Var}\bigg(\sum_{l=1}^n \mathbb I(X_{jl}\leq \bar z_j)\bigg)}. 
\label{opt_cj}
\end{align}

We can show that the CV estimator is unbiased and achieves variance reductions in the following remarks. The reduction in variance, subsequently, allows us to reduce the sample size to achieve the same level of $\delta$ and $\alpha$.

\begin{remark}[Unbiasedness of CV estimator]
The CV estimator \eqref{cv_estimator} is unbiased for $p_j$.
\end{remark}
\begin{proof}
The proof is straightforward, showing $\mathbb E[\hat{p}_j] = p_j$.
\begin{align}
\mathbb E[\hat p_j] =& \frac{1}{n}\sum_{l=1}^n\mathbb E[\mathbb I(Z_{jl}\leq z_j)] \nonumber \\
&+\pi_j \bigg(\frac{1}{n}\sum_{l=1}^n\mathbb E[ \mathbb I(X_{jl}\leq \bar z_j)]-g(\bar z_j) \bigg) \\
=& \frac{1}{n}\sum_{l=1}^np_j + \pi_j \bigg(\frac{1}{n}\sum_{l=1}^n g(\bar z_j) \bigg) - \pi_j g(\bar z_j) \\
=& p_j.
\end{align}
\end{proof}

\begin{remark}[Variance Reduction of CV Estimator]
Assuming we can generate highly correlated random variables $Z_{jl}$ and $X_{jl}$ simultaneously and choose the optimal $\pi_j$ \eqref{opt_cj}, the CV estimator \eqref{cv_estimator} attains a variance reduction. 
\end{remark}
\begin{proof}
Note that the variance without using CV is
\begin{align}
\text{Var}(\hat p_j) = \frac{1}{n^2}\text{Var}\bigg(\sum_{l=1}^n\mathbb I(Z_{jl}\leq z_j)\bigg).
\end{align}
With CV, the variance of the estimator is
\begin{align}
\text{Var}(\hat p_j) = &\frac{1}{n^2} \Bigg( \text{Var}\bigg(\sum_{l=1}^n\mathbb I(Z_{jl}\leq z_j)\bigg) \\
&+2\pi_j \text{Cov} \bigg(\sum_{l=1}^n\mathbb I(Z_{jl}\leq z_j),\sum_{l=1}^n\mathbb I(X_{jl}\leq \bar z_j) \bigg) \nonumber \\
&+\pi_j^2 \text{Var}\bigg(\sum_{l=1}^n\mathbb I(X_{jl}\leq \bar z_j)\bigg) \Bigg) \nonumber .
\end{align}
Plugging in the optimal $\pi_j$ for our problem and simplifying, we have
\begin{align}
\text{Var}(\hat p_j) =& \frac{1}{n^2}  \text{Var}\bigg(\sum_{l=1}^n\mathbb I(Z_{jl}\leq z_j)\bigg) \\
&- \frac{\text{Cov}^2\bigg(\sum_{l=1}^n\ \mathbb I(Z_{jl}\leq z_j), \sum_{l=1}^n\ \mathbb I(X_{jl}\leq \bar z_j) \bigg)}{n^2 \text{Var}\bigg(\sum_{l=1}^n\ \mathbb I(X_{jl}\leq \bar z_j)\bigg)}  \nonumber.
\end{align}
We can see that the second term in RHS is non-positive, which means that the variance is reduced the most if $\mathbb I(Z_{jl} \leq z_j)$ and $\mathbb I(X_{jl} \leq \bar z_j)$ are highly correlated (either positively or negatively), which intuitively means $X_{jl}$ provides some information about $Z_{jl}$. It is important to note, however, that in practice, we often use sample covariances and sample variances to compute $\pi_j$, so the CV estimator might not achieve this theoretical variance reduction. 
\end{proof}


\section{NUMERICAL EXPERIMENTS}
\label{sec:exp}



In this study, we examine the EV and electricity data obtained from Surabaya, Indonesia. The EV dataset includes 11 candidate charging stations, 31 sub-regions of the city representing demand nodes, and two vehicle types, namely motorcycles ($k=1$) and cars ($k=2$). Figure \ref{fig:candidate_location} illustrates the locations of the candidate charging stations (red nodes) and demand nodes (blue nodes), where the size of the blue nodes denotes the size of the demand at each location. This charging demand, i.e. the number of EVs of type $k$ at each demand node $i$, is represented by $w_i^k$. The average travel time from demand node $i$ to charging station $j$ using vehicle $k, d_{ij}^k,$ is amassed from Google Maps.  The full capacity for each charging connector is considered as $c_j=1440$ minutes/day for all $j \in J$ with 24/7 operational hours and the number of connectors installed in station $j \in J$ is limited to $q_j=8$ for all $j \in J$, due to land availability in the candidate locations.

We estimate the disruption probability by simulating random electricity demands $Z = [Z_j]_{\forall j \in J}$ where $Z_j \sim q_j$. We obtain this masked data from the local electricity company, which performed data masking and rescaling for privacy and security reasons. The masked mean and standard deviation of $q_j$ along with demand threshold $z_j$ are summarized in Table \ref{tab:masked_demand_model}. The simulation uses this probability model to generate random demands and an electricity disruption event is triggered for the whole day at station $j$ when $Z_j  \geq z_j$. Hence, we have station reliability $p_j = \mathbb P(Z_j \leq z_j), \forall j \in J$. The other experiment parameters are summarized in Table \ref{tab:model_params}.

\begin{table*}
\centering
\begin{threeparttable}
\caption{The mean and standard deviation of the masked cumulative electricity demand probability $q_j = N(\mu_j, \sigma^2_j)$ and its threshold $z_j, \forall j \in J$}

\begin{tabular}{@{} c S[table-format=5.0] S[table-format=5.0] S[table-format=5.0] @{}} 
\toprule
{Station \# ($j$)} & {Mean ($\mu_j$)} & {Standard Deviation ($\sigma_j$)} & {Threshold ($z_j$)} \\ 
\midrule
  1 &   28979 & 3622 & 36224  \\
  2 & 11590 & 1440 & 14490  \\
  3 & 5795 & 722 & 7250  \\
  4 & 5789 & 715 & 7245   \\
  5 & 17387 & 2180 & 21750   \\
  6 & 17385 & 2169 & 21730   \\
  7 & 11590 & 1442 & 14475   \\
  8 & 11595 & 1435 & 14481   \\
  9 & 5797 & 720 & 7246   \\
  10 & 5785 & 731 & 7241   \\
  11 & 11600 & 1450 & 14500   \\
  
\bottomrule
\end{tabular}

\label{tab:masked_demand_model}
\end{threeparttable}
\end{table*} 


\begin{table*}
\centering
\begin{threeparttable}
\caption{Additional Model Parameters}

\begin{tabular}{@{} c l c S[table-format=5.0] c @{}} 
\toprule
{No} & {Parameter name} & {Symbol} & {Value} & {Unit} \\ 
\midrule
  1 &   \text{Station \#1 operational + investment cost} & $h_1$ & 4602739   & \text{IDR} \\
  ~ & \text{Station \#2 operational + investment cost} & $h_2$ & 1586301   & \text{IDR} \\
  ~ & \text{Station \#3 operational + investment cost} & $h_3$ & 1068493   & \text{IDR} \\
  ~ & \text{Station \#4 operational + investment cost} & $h_4$ & 1150684   & \text{IDR} \\
  ~ & \text{Station \#5 operational + investment cost} & $h_5$ & 1068493   & \text{IDR} \\
   & \text{Station \#6 operational + investment cost} & $h_6$ & 1586301   & \text{IDR} \\
   & \text{Station \#7 operational + investment cost} & $h_7$ & 2794520   & \text{IDR} \\
   & \text{Station \#8 operational + investment cost} & $h_8$ & 1972602   & \text{IDR} \\
   & \text{Station \#9 operational + investment cost} & $h_9$ & 2054794   & \text{IDR} \\
   & \text{Station \#10 operational + investment cost} & $h_{10}$ & 1643835   & \text{IDR} \\
   & \text{Station \#11 operational + investment cost} & $h_{11}$ & 1643835   & \text{IDR} \\
  2 & \text{Electricity pricing rate}   & $r$ & 2467  & \text{IDR/kWh} \\ 
  3 & \text{Penalty rate for unmet demand}   & $s$ & 50000 & \text{IDR/minute/vehicle} \\
  4 & \text{Full-charging power need for motorcycle}  & $e_1$ & 90  & \text{kWh} \\
   & \text{Full-charging power need for car}  & $e_2$ & 133  & \text{kWh} \\
  5 & \text{Full-charging time for motorcycle}  & $t_1$ & 20  & \text{minute} \\
   & \text{Full-charging time for car}  & $t_2$ & 39  & \text{minute} \\
  6 & \text{Charging connector operational cost}    & $g$ & 479285   & \text{IDR/unit} \\
  7 & \text{Max number of alternative candidates}   & $N$ & 11 & \text{station} \\
  8 & \text{Max travel time threshold}   & $d_{max}$ & 35  & \text{minutes} \\
  9 & \text{Min service level threshold}   & $\bar p$ & 0.95  & \text{n/a} \\
  10 & \text{Big-M}   & $M$ & 99999999  & \text{n/a} \\
  
\bottomrule
\end{tabular}

\label{tab:model_params}
\end{threeparttable}
\end{table*} 



We then build our model by running $n$ simulation replications and computing the mean of the objective function values. The result is summarized in Fig. \ref{fig:cv_est} and Fig. \ref{fig:cv_std} for $n$ up 
 to 10,000. The selected stations and demand assignments for each model solution are shown in Fig. \ref{fig:nrm_rm} (left: Non-Robust Model, right: Robust Model) and Fig. \ref{fig:miss1_miss2} (left: Misspecified Model \#1, right: Misspecified Model \#2). The Misspecified Model \#1 is built assuming $0.95p_j$ while the Misspecified Model \#2 assumes $1.05p_j$ for all $j \in J$, highlighting underestimation and overestimation of service reliability respectively. 
 
 The CV estimator is constructed using standard normal random variables $X_{jl}$ with $\bar z_j$ properly scaled. This gives a highly correlated random variables $\mathbb I(X_{jl} \leq \bar z_j)$ to $\mathbb I(Z_{jl} \leq z_j)$. We show the estimated station reliability ($p_j$) using MC and CV in Fig. \ref{fig:estimation} and its standard error in Fig. \ref{fig:estimation_error} to highlight the superior estimation efficiency using the CV estimator. 

\begin{figure*}
    \centering
    \includegraphics[width=0.25\linewidth]{figs/oscm.png}    
    \caption{The summary of simulated objective function values. The thick line denotes the mean given the sample size $n$ and the shaded area denotes the 95\% confidence interval.  The efficiency-enhanced simulation using CV clearly provides a much tighter confidence interval compared to MC for any sample size $n$.
    }
    \label{fig:cv_est}
\end{figure*}

\begin{figure*}
    \centering
    \includegraphics[width=0.25\linewidth]{figs/oscm.png}
    \caption{The standard error of simulated objective function values over sample size $n$. The standard error of the CV estimates is about 10$\times$ smaller than MC estimates, suggesting the CV's superior efficiency.
    }
    \label{fig:cv_std}
\end{figure*}

\begin{figure*}
    \centering   
    \includegraphics[width=0.25\linewidth]{figs/oscm.png}
    \includegraphics[width=0.25\linewidth]{figs/oscm.png}
    \caption{The optimal solutions of the Non-Robust Model (left figure) and Robust Model (right figure).}
    \label{fig:nrm_rm}
\end{figure*}

\begin{figure*}
    \centering   
    \includegraphics[width=0.25\linewidth]{figs/oscm.png}~~\includegraphics[width=0.25\linewidth]{figs/oscm.png}
    \caption{The optimal solutions of the Misspecified Model \#1 (left figure) and Misspecified Model \#2 (right figure).}
    \label{fig:miss1_miss2}
\end{figure*}



\begin{figure*}
    \centering
    \includegraphics[width=0.25\linewidth]{figs/oscm.png}
    \caption{The estimated reliability for each of the 11 candidate charging stations using MC estimator (orange line) and CV estimator (blue line). The shaded region represents the 95\% confidence interval of the estimator given sample size $n$. Again, clearly, CV estimates produce tighter confidence intervals, highlighting its effectiveness in enhancing the simulation efficiency, allowing terminating with fewer sample size $n$ while achieving the same (or even better) confidence level vs MC.}
    \label{fig:estimation}
\end{figure*}

\begin{figure*}
    \centering
    \includegraphics[width=0.25\linewidth]{figs/oscm.png}
    \caption{The standard error of the estimated station reliability in Fig. \ref{fig:estimation} from MC (orange line) and CV (blue line). Overall, CV produces an estimate with 10$\times$ higher precision, compared to MC, allowing our proposed method to handle cases when disruptions occur more rarely (e.g., in cities in a developed country).}
    \label{fig:estimation_error}
\end{figure*}



\section{DISCUSSION}
\label{sec:discussion}

In this section, we discuss our findings regarding the robustness of the optimal solutions against disruptions even when the probability is misspecified and the enhanced disruption simulation efficiency that allows robust decision-making for our problem against disruption uncertainties. We also highlight the limitation of the model and our outlook for future research.

\subsection{Robustness of the Optimal Solutions}


Figure \ref{fig:cv_est} summarizes the objective function values obtained by benchmarking the Robust Model, Non-Robust Model, Misspecified Model \#1 (underestimated station reliability), and Misspecified Model \#2 (overestimated station reliability). The optimal solution of the Robust Model (represented by orange and brown lines) outperforms the other models. Conversely, the solution of the Non-Robust Model (represented by blue and purple lines) yields the lowest objective value. The Non-Robust Model prioritizes minimizing operational and investment costs, resulting in only two charging stations being opened. This leads to lower revenue and higher penalties, particularly during disruptions. In contrast, the Robust Model balances operational and investment costs with potential revenue losses and penalties incurred during disruptions. As a result, the Robust Model opens three charging stations, distributing the large charging stations across the geography of the city, resulting in an 18\% higher total cost than the Non-Robust Model solution. However, it provides better protection against revenue loss and penalties incurred during disruptions. We also suggest that these charging stations implement a smart energy management policy \autocite{li2020energy} for added robustness. This added robustness leads to a 10\% higher revenue and 60\% lower penalty when disruptions occur, yielding an approximately 13\% higher overall objective. Figure \ref{fig:nrm_rm} shows that the Robust Model's balanced solution covers more demand points with two charging stations, resulting in a better revenue and penalty trade-off than the Non-Robust Model.


The Robust Model with misspecified station reliability still provides some level of robustness, as evidenced by the objective values of both the underestimation and overestimation scenarios. These models' solutions have objective values lower than the Robust Model solution but higher than the Non-Robust Model solution. Thus, while accurately estimating station reliability is beneficial, the model can still tolerate imperfections. When utilizing the Robust Model with underestimated station reliability, the solution tends to be more conservative and provides a higher level of buffer against disruptions. This results in a solution with four charging stations, with over 90\% of demand points covered by two or more charging stations. On the other hand, overestimating station reliability leads to a solution with only three charging stations, resulting in a lower cost and an objective value very close to the Robust Model. Figure \ref{fig:miss1_miss2} illustrates the charging station placement for both the underestimated and overestimated scenarios.

\subsection{Improved Simulation Efficiency using CV Estimator}

We now discuss how we incorporate the simulation into our robust model. The main challenges center around incorporating electricity station reliability $p_j, \forall j \in J$ (and thus corresponding disruption probability $1-p_j, \forall j \in J$ ), which might require a huge sample size to achieve desired precision level (thus increasing the computational burden of computing the objective function (either \eqref{obj_est_sim} or \eqref{obj_est_phat}) and the reliability constraints (either \eqref{reliability}-\eqref{reliability_v} or \eqref{reliability_est_phat}-\eqref{reliability_v_est_hat}). 

While both MC and CV estimators of the objective values are unbiased and converge to the same value for each model, the proposed CV estimation approach appears to effectively reduce the estimation variance, thus yielding tighter confidence intervals in Fig. \ref{fig:cv_est} (brown, silver, pink, and purple lines vs. orange, red, green, and blue lines). Furthermore, Fig. \ref{fig:cv_std} highlight that all CV estimators attain about 10$\times$ smaller standard errors compared to their MC counterparts. This means that CV improves the simulation efficiency and \textit{reduces the sample size required to attain the same precision up to a factor of 10} vs. naive MC simulation approach, \textit{without accuracy loss}. 


The dominant efficiency performance of the CV-based estimation technique that reduces the sample size requirement while maintaining accuracy allows us to incorporate the estimated station reliability into the objective function and reliability constraints. This results in the proposed Robust Model that can be solved without increasing the computational cost significantly. The high efficiency of the CV over MC  in estimating the reliability probabilities (even to values close to 1.00) is emphasized in Fig. \ref{fig:estimation}, in which all CV estimates attain much tighter confidence intervals regardless of the target probability. In this estimation, again, CV estimators attain 10$\times$ smaller standard error for the same sample size used by MC estimators. This highlights the applicability of our robust modeling method to deal with problems where electricity disruptions are extremely rare and need to be estimated to an ultra-level precision.

\subsection{Limitation of the Current Work}

Although our CV-assisted robust model provides optimal solutions that strike a balance between minimal cost and buffering against electricity disruptions, we acknowledge that scaling it to larger problems, such as a larger charging station candidate set and more fine-grained demand points, heavily relies on the efficiency of the MIP solver. Moreover, we acknowledge that the electricity pricing rate used in this study is simplified, whereas more recent dynamic electricity pricing schemes are available and more realistic, though highly nonlinear. Incorporating such schemes could improve the accuracy of our revenue model, but it may not be feasible with our current solver. Additionally, the CV estimation approach used in this study is based on some prior knowledge about the probability model of the random variable triggering the disruption events. In practice, such knowledge may not be easy to obtain. However, we recognize that machine learning models can be leveraged to extract features from historical datasets and estimate disruption events. We can also leverage machine learning techniques to estimate the battery capacity of the EVs \autocite{zhao2023machine} to better predict the charging time for each arriving demand to extend our model to incorporate nonlinear dynamics and more realistic operations in our future work.

\section{CONCLUSION}
\label{sec:conclusion}

In this study, we propose a simulation-based optimization model to address the critical issue of designing robust planning for EV charging stations in developing countries, where electricity disruptions may frequently occur and impact customer satisfaction. Our model considers service reliability as a key factor and incorporates it into the objective function and constraints using the control variates (CV) variance reduction technique to improve simulation efficiency. Our numerical experiment, based on a dataset from Surabaya, Indonesia, demonstrates the superior performance of our robust model solution compared to its non-robust counterpart, even in cases of underestimated or overestimated service reliability. While our proposed model shows promise, we acknowledge its reliance on an efficient MIP solver and its use of a simplified electricity pricing rate. Furthermore, our CV estimator is based on prior knowledge of the probability model, which may not be available in practice. As such, we seek to extend our model to cover nonlinear MIP and learning-based disruption estimation in future work. Nonetheless, our model's ability to reduce the required sample size by up to 10$\times$ compared to Monte Carlo simulations highlights its potential to provide a robust solution to the challenges associated with EV charging infrastructure under random electricity disruptions.


\printbibliography[title={REFERENCES}]

\biographies
\begin{biography}{Author Name}
performance of our robust model solution compared to its non-robust counterpart, even in cases of underestimated or overestimated service reliability. While our proposed model shows promise, we acknowledge its reliance on an efficient MIP solver and its use of a simplified electricity pricing rate. Furthermore, our CV estimator is based on prior knowledge of the probability model, which may not be available in practice. As such, we seek to extend our model to cover nonlinear MIP and learning-based disruption estimation in future work. Nonetheless, our model's ability to reduce the required sample size by up to 10$\times$ compared to Monte Carlo simulations highlights its potential to provide a robust solution to the challenges associated with EV charging infrastructure under random electricity disruptions.
\end{biography}

\begin{biography}{Author Name}
performance of our robust model solution compared to its non-robust counterpart, even in cases of underestimated or overestimated service reliability. While our proposed model shows promise, we acknowledge its reliance on an efficient MIP solver and its use of a simplified electricity pricing rate. Furthermore, our CV estimator is based on prior knowledge of the probability model, which may not be available in practice. As such, we seek to extend our model to cover nonlinear MIP and learning-based disruption estimation in future work. Nonetheless, our model's ability to reduce the required sample size by up to 10$\times$ compared to Monte Carlo simulations highlights its potential to provide a robust solution to the challenges associated with EV charging infrastructure under random electricity disruptions.
\end{biography}

\begin{biography}{Author Name}
performance of our robust model solution compared to its non-robust counterpart, even in cases of underestimated or overestimated service reliability. While our proposed model shows promise, we acknowledge its reliance on an efficient MIP solver and its use of a simplified electricity pricing rate. Furthermore, our CV estimator is based on prior knowledge of the probability model, which may not be available in practice. As such, we seek to extend our model to cover nonlinear MIP and learning-based disruption estimation in future work. Nonetheless, our model's ability to reduce the required sample size by up to 10$\times$ compared to Monte Carlo simulations highlights its potential to provide a robust solution to the challenges associated with EV charging infrastructure under random electricity disruptions.
\end{biography}

\end{document}
