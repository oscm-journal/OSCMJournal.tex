\documentclass{oscmjournal}
\usepackage[utf8]{inputenc}
\usepackage[english]{babel}
\usepackage{graphicx}
\usepackage{hyperref}



\title{A Historical Review on Omni Channel Retailing Consumer Research}
\authorinfo{
    \textbf{Author 1}\\
    University of Tennessee, Knoxville\\
    Haslam College of Business\\
    Email: email@email.com\\
    \vspace{1em}
    \textbf{Author 2}\\
    University of Kansas, Knoxville\\
    Something College of Business\\
    Email: email2@email.com
}
\keywords{omnichannel, consumer, retailing, review, bibliometric}


\begin{document}

\maketitle

\begin{abstract}
    Consumer research is crucial in omnichannel retailing and is considered a primary focus in distribution management. This article presents an objective, systematic, and comprehensive review of the current literature on omnichannel consumers. We aim to systematically map the knowledge of omnichannel consumer research's thematic structure, theory, and methodology. We employed an integrated approach to analyze 152 journal articles, using bibliometric methods, including country and citation analysis, co-citation networks, literature coupling, thematic mapping, and historical citation networks. We conducted an exploratory analysis of selected studies to examine the features of countries, journals, authors, highly cited literature, and trends in this field. Additionally, the study underwent bibliometric analyses of keyword co-occurrence, bibliographic coupling, historical citations, and co-citation analysis. The keyword co-occurrence analysis was based on the frequency and importance of keywords, identifying four distinct clusters. The findings reveal that omnichannel research is a growing field. Scholars have examined omnichannel consumers from multiple perspectives, utilizing different research methods and theoretical foundations. However, further research is needed to explore omnichannel consumers in more countries and address factors such as regional and cultural differences. This research employs advanced tools for the first time to review the literature on omnichannel consumers and provides a comprehensive view of the topics of importance discussed in the literature on omnichannel consumer management. This study will provide insights for practitioners and academic researchers to improve decision-making and develop strategies.
\end{abstract}


\section{Introduction}

With the end of the pandemic and the surge in consumer demand, retailers and consumer brands face significant macroeconomic challenges, including inflation, labor shortages, and disruptions in the supply chain (WEF, 2022) \citep{queiroz2022supply}. These uncertainties have left retailers needing to evaluate their future profitability. While some forward-thinking retailers have already embraced omnichannel commerce models (e.g., Amazon, Walmart, and Starbucks), the speed at which consumer demand changes is increasing (Adivar et al., 2019). Companies must embrace the evolving landscape, where customers expect a seamless shopping experience (Verhoef et al., 2018; Liu et al., 2022) and have the ability to shop, access information, share ideas, and seek assistance through various channels (Jindal et al., 2021).

The Monty Hall problem is an interesting puzzle in mathematics inspired by the TV  game show "Let's make a deal". It's a famous problem that is really easy to understand. The problem became famous after its appearance in the column "Ask Marilyn" in 1990 and  it's described below: Suppose you're on a game show, and you're given the choice of three doors: Behind one  door is a car; behind the others, goats. You pick a door, say No. 1, and the host, who  knows what's behind the doors, opens another door, say No. 3, which has a goat. He then  says to you, "Do you want to pick door No. 2?" Is it to your advantage to switch your choice?


Let's try to figure it out. At this point there are two doors, behind one door is a goat  and behind the other one is a car, thus the probability of choosing the right one is 50\%, changing the decision doesn't make any difference in the chances of winning. It's pretty much the same scenario of tossing a coin. Even though the ideas exposed in the previous paragraph may seem correct, actually switching the choice increases the probability of winning. To understand why let's  first check the facts:

\section{The same thing}

\subsection{Subsection}

The Monty Hall problem is an interesting puzzle in mathematics inspired by the TV  game show "Let's make a deal". It's a famous problem that is really easy to understand. The problem became famous after its appearance in the column "Ask Marilyn" in 1990 and  it's described below: Suppose you're on a game show, and you're given the choice of three doors: Behind one  door is a car; behind the others, goats. You pick a door, say No. 1, and the host, who  knows what's behind the doors, opens another door, say No. 3, which has a goat. He then  says to you, "Do you want to pick door No. 2?" Is it to your advantage to switch your choice?
\subsubsection{Subsubsection}

Let's try to figure it out. At this point there are two doors, behind one door is a goat  and behind the other one is a car, thus the probability of choosing the right one is 50\%, changing the decision doesn't make any difference in the chances of winning. It's pretty much the same scenario of tossing a coin. Even though the ideas exposed in the previous paragraph may seem correct, actually switching the choice increases the probability of winning. To understand why let's  first check the facts.

\subsection{Subsection}
The Monty Hall problem is an interesting puzzle in mathematics inspired by the TV  game show "Let's make a deal". It's a famous problem that is really easy to understand. The problem became famous after its appearance in the column "Ask Marilyn" in 1990 and  it's described below: Suppose you're on a game show, and you're given the choice of three doors: Behind one  door is a car; behind the others, goats. You pick a door, say No. 1, and the host, who  knows what's behind the doors, opens another door, say No. 3, which has a goat. He then  says to you, "Do you want to pick door No. 2?" Is it to your advantage to switch your choice?

Let's try to figure it out. At this point there are two doors, behind one door is a goat  and behind the other one is a car, thus the probability of choosing the right one is 50\%, changing the decision doesn't make any difference in the chances of winning. It's pretty much the same scenario of tossing a coin. Even though the ideas exposed in the previous paragraph may seem correct, actually switching the choice increases the probability of winning. To understand why let's  first check the facts.

The Monty Hall problem is an interesting puzzle in mathematics inspired by the TV  game show "Let's make a deal". It's a famous problem that is really easy to understand. The problem became famous after its appearance in the column "Ask Marilyn" in 1990 and  it's described below: Suppose you're on a game show, and you're given the choice of three doors: Behind one  door is a car; behind the others, goats. You pick a door, say No. 1, and the host, who  knows what's behind the doors, opens another door, say No. 3, which has a goat. He then  says to you, "Do you want to pick door No. 2?" Is it to your advantage to switch your choice?

Let's try to figure it out. At this point there are two doors, behind one door is a goat  and behind the other one is a car, thus the probability of choosing the right one is 50\%, changing the decision doesn't make any difference in the chances of winning. It's pretty much the same scenario of tossing a coin. Even though the ideas exposed in the previous paragraph may seem correct, actually switching the choice increases the probability of winning. To understand why let's  first check the facts.

\section{Introduction}

The Monty Hall problem is an interesting puzzle in mathematics inspired by the TV  game show "Let's make a deal". It's a famous problem that is really easy to understand. The problem became famous after its appearance in the column "Ask Marilyn" in 1990 and  it's described below: Suppose you're on a game show, and you're given the choice of three doors: Behind one  door is a car; behind the others, goats. You pick a door, say No. 1, and the host, who  knows what's behind the doors, opens another door, say No. 3, which has a goat. He then  says to you, "Do you want to pick door No. 2?" Is it to your advantage to switch your choice?


Let's try to figure it out. At this point there are two doors, behind one door is a goat  and behind the other one is a car, thus the probability of choosing the right one is 50\%, changing the decision doesn't make any difference in the chances of winning. It's pretty much the same scenario of tossing a coin. Even though the ideas exposed in the previous paragraph may seem correct, actually switching the choice increases the probability of winning. To understand why let's  first check the facts:

\section{The same thing}

The Monty Hall problem is an interesting puzzle in mathematics inspired by the TV  game show "Let's make a deal". It's a famous problem that is really easy to understand. The problem became famous after its appearance in the column "Ask Marilyn" in 1990 and  it's described below: Suppose you're on a game show, and you're given the choice of three doors: Behind one  door is a car; behind the others, goats. You pick a door, say No. 1, and the host, who  knows what's behind the doors, opens another door, say No. 3, which has a goat. He then  says to you, "Do you want to pick door No. 2?" Is it to your advantage to switch your choice?

Let's try to figure it out. At this point there are two doors, behind one door is a goat  and behind the other one is a car, thus the probability of choosing the right one is 50\%, changing the decision doesn't make any difference in the chances of winning. It's pretty much the same scenario of tossing a coin. Even though the ideas exposed in the previous paragraph may seem correct, actually switching the choice increases the probability of winning. To understand why let's  first check the facts.

The Monty Hall problem is an interesting puzzle in mathematics inspired by the TV  game show "Let's make a deal". It's a famous problem that is really easy to understand. The problem became famous after its appearance in the column "Ask Marilyn" in 1990 and  it's described below: Suppose you're on a game show, and you're given the choice of three doors: Behind one  door is a car; behind the others, goats. You pick a door, say No. 1, and the host, who  knows what's behind the doors, opens another door, say No. 3, which has a goat. He then  says to you, "Do you want to pick door No. 2?" Is it to your advantage to switch your choice?

Let's try to figure it out. At this point there are two doors, behind one door is a goat  and behind the other one is a car, thus the probability of choosing the right one is 50\%, changing the decision doesn't make any difference in the chances of winning. It's pretty much the same scenario of tossing a coin. Even though the ideas exposed in the previous paragraph may seem correct, actually switching the choice increases the probability of winning. To understand why let's  first check the facts.

The Monty Hall problem is an interesting puzzle in mathematics inspired by the TV  game show "Let's make a deal". It's a famous problem that is really easy to understand. The problem became famous after its appearance in the column "Ask Marilyn" in 1990 and  it's described below: Suppose you're on a game show, and you're given the choice of three doors: Behind one  door is a car; behind the others, goats. You pick a door, say No. 1, and the host, who  knows what's behind the doors, opens another door, say No. 3, which has a goat. He then  says to you, "Do you want to pick door No. 2?" Is it to your advantage to switch your choice?

Let's try to figure it out. At this point there are two doors, behind one door is a goat  and behind the other one is a car, thus the probability of choosing the right one is 50\%, changing the decision doesn't make any difference in the chances of winning. It's pretty much the same scenario of tossing a coin. Even though the ideas exposed in the previous paragraph may seem correct, actually switching the choice increases the probability of winning. To understand why let's  first check the facts.

\section{The same thing}

The Monty Hall problem is an interesting puzzle in mathematics inspired by the TV  game show "Let's make a deal". It's a famous problem that is really easy to understand. The problem became famous after its appearance in the column "Ask Marilyn" in 1990 and  it's described below: Suppose you're on a game show, and you're given the choice of three doors: Behind one  door is a car; behind the others, goats. You pick a door, say No. 1, and the host, who  knows what's behind the doors, opens another door, say No. 3, which has a goat. He then  says to you, "Do you want to pick door No. 2?" Is it to your advantage to switch your choice?

Let's try to figure it out. At this point there are two doors, behind one door is a goat  and behind the other one is a car, thus the probability of choosing the right one is 50\%, changing the decision doesn't make any difference in the chances of winning. It's pretty much the same scenario of tossing a coin. Even though the ideas exposed in the previous paragraph may seem correct, actually switching the choice increases the probability of winning. To understand why let's  first check the facts.

The Monty Hall problem is an interesting puzzle in mathematics inspired by the TV  game show "Let's make a deal". It's a famous problem that is really easy to understand. The problem became famous after its appearance in the column "Ask Marilyn" in 1990 and  it's described below: Suppose you're on a game show, and you're given the choice of three doors: Behind one  door is a car; behind the others, goats. You pick a door, say No. 1, and the host, who  knows what's behind the doors, opens another door, say No. 3, which has a goat. He then  says to you, "Do you want to pick door No. 2?" Is it to your advantage to switch your choice?

Let's try to figure it out. At this point there are two doors, behind one door is a goat  and behind the other one is a car, thus the probability of choosing the right one is 50\%, changing the decision doesn't make any difference in the chances of winning. It's pretty much the same scenario of tossing a coin. Even though the ideas exposed in the previous paragraph may seem correct, actually switching the choice increases the probability of winning. To understand why let's  first check the facts.

The Monty Hall problem is an interesting puzzle in mathematics inspired by the TV  game show "Let's make a deal". It's a famous problem that is really easy to understand. The problem became famous after its appearance in the column "Ask Marilyn" in 1990 and  it's described below: Suppose you're on a game show, and you're given the choice of three doors: Behind one  door is a car; behind the others, goats. You pick a door, say No. 1, and the host, who  knows what's behind the doors, opens another door, say No. 3, which has a goat. He then  says to you, "Do you want to pick door No. 2?" Is it to your advantage to switch your choice?

Let's try to figure it out. At this point there are two doors, behind one door is a goat  and behind the other one is a car, thus the probability of choosing the right one is 50\%, changing the decision doesn't make any difference in the chances of winning. It's pretty much the same scenario of tossing a coin. Even though the ideas exposed in the previous paragraph may seem correct, actually switching the choice increases the probability of winning. To understand why let's  first check the facts.


\section{The same thing}

The Monty Hall problem is an interesting puzzle in mathematics inspired by the TV  game show "Let's make a deal". It's a famous problem that is really easy to understand. The problem became famous after its appearance in the column "Ask Marilyn" in 1990 and  it's described below: Suppose you're on a game show, and you're given the choice of three doors: Behind one  door is a car; behind the others, goats. You pick a door, say No. 1, and the host, who  knows what's behind the doors, opens another door, say No. 3, which has a goat. He then  says to you, "Do you want to pick door No. 2?" Is it to your advantage to switch your choice?

Let's try to figure it out. At this point there are two doors, behind one door is a goat  and behind the other one is a car, thus the probability of choosing the right one is 50\%, changing the decision doesn't make any difference in the chances of winning. It's pretty much the same scenario of tossing a coin. Even though the ideas exposed in the previous paragraph may seem correct, actually switching the choice increases the probability of winning. To understand why let's  first check the facts.

The Monty Hall problem is an interesting puzzle in mathematics inspired by the TV  game show "Let's make a deal". It's a famous problem that is really easy to understand. The problem became famous after its appearance in the column "Ask Marilyn" in 1990 and  it's described below: Suppose you're on a game show, and you're given the choice of three doors: Behind one  door is a car; behind the others, goats. You pick a door, say No. 1, and the host, who  knows what's behind the doors, opens another door, say No. 3, which has a goat. He then  says to you, "Do you want to pick door No. 2?" Is it to your advantage to switch your choice?

Let's try to figure it out. At this point there are two doors, behind one door is a goat  and behind the other one is a car, thus the probability of choosing the right one is 50\%, changing the decision doesn't make any difference in the chances of winning. It's pretty much the same scenario of tossing a coin. Even though the ideas exposed in the previous paragraph may seem correct, actually switching the choice increases the probability of winning. To understand why let's  first check the facts.

The Monty Hall problem is an interesting puzzle in mathematics inspired by the TV  game show "Let's make a deal". It's a famous problem that is really easy to understand. The problem became famous after its appearance in the column "Ask Marilyn" in 1990 and  it's described below: Suppose you're on a game show, and you're given the choice of three doors: Behind one  door is a car; behind the others, goats. You pick a door, say No. 1, and the host, who  knows what's behind the doors, opens another door, say No. 3, which has a goat. He then  says to you, "Do you want to pick door No. 2?" Is it to your advantage to switch your choice?

Let's try to figure it out. At this point there are two doors, behind one door is a goat  and behind the other one is a car, thus the probability of choosing the right one is 50\%, changing the decision doesn't make any difference in the chances of winning. It's pretty much the same scenario of tossing a coin. Even though the ideas exposed in the previous paragraph may seem correct, actually switching the choice increases the probability of winning. To understand why let's  first check the facts.


\section{The same thing}

The Monty Hall problem is an interesting puzzle in mathematics inspired by the TV  game show "Let's make a deal". It's a famous problem that is really easy to understand. The problem became famous after its appearance in the column "Ask Marilyn" in 1990 and  it's described below: Suppose you're on a game show, and you're given the choice of three doors: Behind one  door is a car; behind the others, goats. You pick a door, say No. 1, and the host, who  knows what's behind the doors, opens another door, say No. 3, which has a goat. He then  says to you, "Do you want to pick door No. 2?" Is it to your advantage to switch your choice?

Let's try to figure it out. At this point there are two doors, behind one door is a goat  and behind the other one is a car, thus the probability of choosing the right one is 50\%, changing the decision doesn't make any difference in the chances of winning. It's pretty much the same scenario of tossing a coin. Even though the ideas exposed in the previous paragraph may seem correct, actually switching the choice increases the probability of winning. To understand why let's  first check the facts.

The Monty Hall problem is an interesting puzzle in mathematics inspired by the TV  game show "Let's make a deal". It's a famous problem that is really easy to understand. The problem became famous after its appearance in the column "Ask Marilyn" in 1990 and  it's described below: Suppose you're on a game show, and you're given the choice of three doors: Behind one  door is a car; behind the others, goats. You pick a door, say No. 1, and the host, who  knows what's behind the doors, opens another door, say No. 3, which has a goat. He then  says to you, "Do you want to pick door No. 2?" Is it to your advantage to switch your choice?

Let's try to figure it out. At this point there are two doors, behind one door is a goat  and behind the other one is a car, thus the probability of choosing the right one is 50\%, changing the decision doesn't make any difference in the chances of winning. It's pretty much the same scenario of tossing a coin. Even though the ideas exposed in the previous paragraph may seem correct, actually switching the choice increases the probability of winning. To understand why let's  first check the facts.

The Monty Hall problem is an interesting puzzle in mathematics inspired by the TV  game show "Let's make a deal". It's a famous problem that is really easy to understand. The problem became famous after its appearance in the column "Ask Marilyn" in 1990 and  it's described below: Suppose you're on a game show, and you're given the choice of three doors: Behind one  door is a car; behind the others, goats. You pick a door, say No. 1, and the host, who  knows what's behind the doors, opens another door, say No. 3, which has a goat. He then  says to you, "Do you want to pick door No. 2?" Is it to your advantage to switch your choice?

Let's try to figure it out. At this point there are two doors, behind one door is a goat  and behind the other one is a car, thus the probability of choosing the right one is 50\%, changing the decision doesn't make any difference in the chances of winning. It's pretty much the same scenario of tossing a coin. Even though the ideas exposed in the previous paragraph may seem correct, actually switching the choice increases the probability of winning. To understand why let's  first check the facts.


\bibliographystyle{apalike}
\bibliography{references}

\end{document}
